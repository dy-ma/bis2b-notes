\documentclass[12pt]{article}
\usepackage[utf8]{inputenc}
\usepackage[backend=biber]{biblatex}
\usepackage{tikz}
\usepackage[hypcap=false]{caption}
\usepackage{amsmath}
\usepackage[hidelinks]{hyperref}

\usetikzlibrary{fit,shapes.geometric}

% Code for circling tabular cells
\newcounter{nodemarkers}
\newcommand\circletext[1]{%
    \tikz[overlay,remember picture] 
        \node (marker-\arabic{nodemarkers}-a) at (0,1.5ex) {};%
    #1%
    \tikz[overlay,remember picture]
        \node (marker-\arabic{nodemarkers}-b) at (0,0){};%
    \tikz[overlay,remember picture,inner sep=2pt]
        \node[draw,ellipse,fit=(marker-\arabic{nodemarkers}-a.center) (marker-\arabic{nodemarkers}-b.center)] {};%
    \stepcounter{nodemarkers}%
}

\title{BIS 2B Math Equations}
\date{May 2021}
\author{Dylan Ang}

\addbibresource{b2b-math.bib}

\begin{document}

\maketitle

\tableofcontents

\section{Linkage Mapping}

Chromosomal crossover, or crossing over, is the exchange of genetic material during sexual reproduction between two homologous chromosomes' non-sister chromatids that results in recombinant chromosomes \cite{crossover}.

\begin{equation}\label{Recombination}
    \text{\% Recombination} = \frac{\text{\# of Recombinant Offspring}}{\text{Total \# of Offspring}}
\end{equation}

Note: $1\% \text{ Recombination} = \text{1 map unit}$

Consider the following sample data, which shows the number of individuals in a sample population with a certain genotype. Note that there are 1000 total individuals.

\begin{table}[ht]
    \centering
    \begin{tabular}{*4{|c}|}
        \hline
        Abc 18 & aBc 112 & abc 308 & ABc 66 \\ \hline
        abC 59 & ABC 320 & AbC 102 & aBC 15 \\ \hline
    \end{tabular}
\end{table}

You may be asked to calculate the distance between two genes $a$ and $b$. To find this distance, note which genotypes feature a crossover event between $a$ and $b$. i.e when $a$ and $b$ do not match case. These are

\begin{itemize}
    \item $\text{Abc } 18$
    \item $\text{aBc } 112$
    \item $\text{AbC } 102$
    \item $\text{aBC } 15$
\end{itemize}

Plug into equation (\ref{Recombination})

$$\frac{18+112+102+15}{1000} = 24.7\% = 24.7 \text{ map units}$$

\subsection{Identifying Gene Order}

In addition to calculating distance between genes, you may be asked to identify the order of the genes.

Consider two sister chromatids,

\begin{center}
    \begin{tikzpicture}
        \filldraw[color=magenta] (0,0) -- (3,0);
        \filldraw[color=cyan] (0,0.5) -- (3,0.5);
    \end{tikzpicture}
    \captionof{figure}{Sister Chromatids}
\end{center}

\vspace{2mm}

\begin{center}
    \begin{tikzpicture}
        \filldraw[color=cyan] (0,0.5) -- (1,0.5);
        \filldraw[color=magenta] (1,0.5) -- (3,0.5);
        \filldraw[color=magenta] (0,0) -- (1,0);
        \filldraw[color=cyan] (1,0) -- (3,0);

        \filldraw[color=magenta] (4,0.5) -- (6,0.5);
        \filldraw[color=cyan] (6,0.5) -- (7,0.5);
        \filldraw[color=cyan] (4,0) -- (6,0);
        \filldraw[color=magenta] (6,0) -- (7,0);
    \end{tikzpicture}
    \captionof{figure}{Single Recombination Event}
\end{center}

\vspace{2mm}

\begin{center}
    \begin{tikzpicture} 
        \filldraw[color=cyan] (0,0) -- (1,0);
        \filldraw[color=magenta] (1,0) -- (2,0);
        \filldraw[color=cyan] (2,0) -- (3,0);

        \filldraw[color=magenta] (0,0.5) -- (1,0.5);
        \filldraw[color=cyan]    (1,0.5) -- (2,0.5);
        \filldraw[color=magenta] (2,0.5) -- (3,0.5);
    \end{tikzpicture}
    \captionof{figure}{Double Recombination Event}\label{double-recomb}
\end{center}

As you can see in Figure \ref{double-recomb}, double recombination events require two "cuts" compared to just one. This makes successful double recombination much more rare than single or no recombination.

If I told you that $Abc$ and $aBC$ were single recombination events, you'd be able to tell that the $A$ gene is the one crossing over. You'd be able to conclude that $A$ is on the left side or right side of the genome. Unfortunately this doesn't help us very much because this means the gene order could be $ABC$ or $ACB$, as well as the inverses of these, $CBA$ and $BCA$. So while it narrows down the order, it doesn't give us the answer.

Now consider if I told you that $Abc$ and $aBC$ represented the double recombination event. You'd know that there must be two genes crossing over (Hence the "double" in double recombination), and these genes must be $B$ and $C$. If $B$ and $C$ are the two genes crossing over, you can see from Figure \ref{double-recomb} that the odd gene out is in the center position. Therefore $A$ must be in the center of the genome. In summary, If you know which genotype corresponds to a Double Recombination Event, then you can find the order of the alleles on the genome.

In order identify which genotype corresponds to a Double Recombination Event, find the two least frequent genotypes. Recall that Double Recombination events are more rare than single or no mutation events, so we know that these two are the Double Recombination events. Now consider the following sample data, I've circled the two least common genotypes.

\begin{table}[ht]
    \centering
    \begin{tabular}{*4{|c}|}
        \hline
        \circletext{Abc 18} & aBc 112 & abc 308 & ABc 66              \\ \hline
        abC 59              & ABC 320 & AbC 102 & \circletext{aBC 15} \\ \hline
    \end{tabular}
\end{table}

We can conclude that $Abc$ and $aBC$ are the Double Recombination events, and thus the $A$ allele is in between $B$ and $C$ on the genome. Note that we technically don't know if the order is $BAC$ or $CAB$, but these are actually considered the same ordering.

\section{Hardy Weinberg}

In population genetics, the Hardy–Weinberg principle, also known as the Hardy–Weinberg equilibrium, model, theorem, or law, states that allele and genotype frequencies in a population will remain constant from generation to generation in the absence of other evolutionary influences. These influences include genetic drift, mate choice, assortative mating, natural selection, sexual selection, mutation, gene flow, meiotic drive, genetic hitchhiking, population bottleneck, founder effect and inbreeding \cite{hardy}.

\subsection{Equations and Notation}

Frequencies are written as $f(\text{The thing we are testing for})$. For example, the frequency of the AA genotype is $f(AA)$, and the frequency of the $A$ allele is written $f(A)$.

\begin{equation}\label{faa}
    f(AA) = \frac{\text{\# of AA individuals}}{\text{\# of individuals (N)}}
\end{equation}

\begin{equation}\label{fa-simple}
    f(A) = \frac{\text{\# of A alleles}}{\text{total \# of alleles (2N)}}
\end{equation}

There are two ways to calculate allele frequencies from given frequencies. The equation on the right hand side requires that you know the total number of individuals, $N$. Whereas the left hand side only requires that you have the frequencies of the genotypes.

\begin{align}
    f(A) &= f(AA) + \frac{f(Aa)}{2} \label{fa-freq} \\
    &= \frac{2*(\text{\# of AA}) + (\text{\# of Aa})}{2N} \label{fa-count}
\end{align}

\begin{table}[h!]
    \centering
    \begin{tabular}{| c | c |}
        \hline
        Variable & Allele           \\\hline
        p        & Dominant Allele  \\
        q        & Recessive Allele \\\hline
    \end{tabular}
    \begin{tabular}{| c | c |}
        \hline
        Variable & Genotype             \\\hline
        $p^2$    & Homozygous Dominant  \\
        $2pq$    & Heterozygous         \\
        $q^2$    & Homozygous Recessive \\\hline
    \end{tabular}
\end{table}

% \pagebreak

\subsection{Example Problem: Counting Fruit Fly Phenotypes}

$A$ and $a$ code for the eye color of the fruit flies. The $A$ allele codes for red eyes and the $a$ allele codes for white eyes. There is incomplete dominance, and heterozygotes $(Aa)$ have pink eyes.

This question asks you to perform analysis of data on fruit fly genotypes. Given the number of each genotype in each generation, can you determine if there is evolution happening? The basic process for solving these problems is 

\begin{enumerate}
    \item Calculate Observed Genotype Frequencies using the data table
    \item Use these values to calculate Allele Frequencies
    \item Use the calculated allele frequencies to predict the next generation's genotype frequencies.
    \item Calculate the observed genotype frequencies of the next generation using the same method as used for the first generation.
\end{enumerate}

\begin{table}[h]
    \centering
    \begin{tabular}{|c | c c c|}
        \hline
              & \# Red eyed flies & \# pink eyed flies & \# white eyed flies \\\hline
        $S_0$ & 83                & 32                 & 106                 \\
        $S_1$ & 85                & 29                 & 100                 \\
        $F_1$ & 8                 & 22                 & 10                  \\\hline
    \end{tabular}
\end{table}

\subsubsection{Step 1}

First, we need to calculate the observed frequencies of the genotypes. To do this, refer to equation (\ref{faa}). The frequency of the different genotypes are

\begin{align*}
    f(\text{Red Eyes}) &= f(AA) = \frac{83}{83+32+106} = 0.376 \\
    f(\text{Pink Eyes}) &=f(Aa) = \frac{32}{83+32+106} = 0.145 \\
    f(\text{White Eyes}) &=f(aa) = \frac{106}{83+32+106} = 0.480
\end{align*}

We have $83+32+106$ in the denominator, as this is $N$, the total number of individuals in the $S_0$ population. From here we can move on to step 2 by using one of the two equations (\ref{fa-freq}) and (\ref{fa-count}) to calculate the allele frequencies of $A$ and $a$. I will show both.

\subsubsection{Step 2}

The first method is with the equation (\ref{fa-freq}). We just found $f(AA)=0.376$ and $f(Aa)=0.145$, so 

$$f(A) = 0.376 + \frac{0.145}{2} = 0.448$$

Finding $f(a)$ is similar, except we replace $f(AA)$ with $f(aa)$.

$$f(a) = 0.480 + \frac{0.145}{2} = 0.552$$

The second method uses equation (\ref{fa-count}) where $N$ is the total number of individuals. From the table, there are 83 $AA$ individuals, $32$ pink eyed individuals, and $106$ white eyed individuals. In total, there are $N=221$ individuals. Therefore

\begin{align*}
    f(A) &= \frac{2*83 + 32}{2*221} = 0.448 \\
    f(a) &= \frac{2*106 + 32}{2*221} = 0.552
\end{align*}

Both methods result in the same answer, the only reason to use one or the other is if you can not use the original counts and only have the frequencies to go off of. Then, it would make sense to use the equation that is based on frequency.

\subsubsection{Step 3}

If the populations are in Hardy Weinberg Equilibrium, then subsequent generations will have expected genotype frequencies matching $f(AA) = p^2$, $f(Aa)=2pq$, $f(aa)=q^2$ where $p=f(A)$ and $q=f(a)$. The expected genotype frequencies for subsequent generations of $S_0$ are

\begin{align*}
    f(AA) &= p^2 = (0.448)^2 \\
    f(Aa) &= 2pq = 2*(0.448)*(0.552) \\
    f(aa) &= q^2 = (0.552)^2
\end{align*}

\subsubsection{Step 4}

Now that we know what we are expecting to see in $S_1$, calculate the observed genotype frequencies out of the data table in order to compare them to your expected values. If there is a significant difference, then this population is not in Hardy Weinberg Equilibrium.



\printbibliography

\end{document}